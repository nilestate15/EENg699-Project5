\documentclass[12pt]{article}

\usepackage[
   paperheight = 11in,
   paperwidth = 8.5in,
   margin = 1in,
   footskip = 0.45in] {geometry} %  custom paper size and margins
\usepackage{graphicx}
\usepackage{float}
\usepackage{amsmath}
\usepackage{bm}
\usepackage{adjustbox}
\usepackage{color} %  custom colors
\usepackage{multicol}
\usepackage{enumitem}
\usepackage{booktabs}
\usepackage{hhline}
\usepackage{titlesec}
\usepackage{hyphenat}
\usepackage{array}
\usepackage{color} %  custom colors
\usepackage{graphicx}
\usepackage{listings, upquote, textcomp} %  "\lstinline{ }" for inline code
\usepackage[
   colorlinks = true,
   linkcolor = darkblue,
   urlcolor = darkblue,
   hypertexnames = false]{hyperref}

\renewcommand{\vec}[1]     {\underline{#1}}
\setlength{\tabcolsep}{10pt}
\def\arraystretch{1.2}
\definecolor{gray}         {rgb}{0.50,0.50,0.50}
\definecolor{darkblue}     {rgb}{0.00,0.00,0.50}
\renewcommand{\labelenumi}{\arabic{enumi}.}
\renewcommand{\labelenumii}{\alph{enumii}.}
\setlength{\parindent}{0em}
\setlength{\parskip}{10pt plus 2pt minus 2pt}
\titlespacing\section{0pt}{10pt plus 2pt minus 2pt}{0pt plus 2pt minus 2pt}
\raggedbottom
\newcolumntype{L}[1]{>{
   \raggedright\let\newline\\\arraybackslash\hspace{0pt}}p{#1}}
\newcolumntype{C}[1]{>{
   \centering\let\newline\\\arraybackslash\hspace{0pt}}p{#1}}
\newcolumntype{R}[1]{>{
   \raggedleft\let\newline\\\arraybackslash\hspace{0pt}}p{#1}}
\lstdefinestyle{lstMat} {
   language = matlab,
   backgroundcolor = {},
   breakatwhitespace = true,
   breaklines = true,
   postbreak = \space,
   breakindent = 4ex,
   showstringspaces = false,
   basicstyle = %
      \lst@ifdisplaystyle\small\fi
      \ttfamily, %  48 chars in 3.5 in
   commentstyle = \color{gray},
   stringstyle = \color{red},
   keywordstyle = \color{blue},
   tabsize = 3,
   upquote = true}
\lstset{style = lstMat} %  Default syntax.

\definecolor{black}        {rgb}{0.00,0.00,0.00}
\definecolor{blue}         {rgb}{0.00,0.00,1.00}
\definecolor{azure}        {rgb}{0.00,0.40,1.00}
\definecolor{cyan}         {rgb}{0.00,0.80,1.00}
\definecolor{aqua}         {rgb}{0.00,0.80,0.50}
\definecolor{green}        {rgb}{0.00,0.60,0.00}
\definecolor{lime}         {rgb}{0.50,0.80,0.00}
\definecolor{yellow}       {rgb}{1.00,0.80,0.00}
\definecolor{orange}       {rgb}{1.00,0.40,0.00}
\definecolor{red}          {rgb}{1.00,0.00,0.00}
\definecolor{pink}         {rgb}{1.00,0.00,0.50}
\definecolor{magenta}      {rgb}{1.00,0.00,1.00}
\definecolor{purple}       {rgb}{0.50,0.00,1.00}
\definecolor{darkblue}     {rgb}{0.00,0.00,0.50}
\definecolor{gray}         {rgb}{0.50,0.50,0.50}
\definecolor{lightgray}    {rgb}{0.80,0.80,0.80}
\definecolor{shade}        {rgb}{0.95,0.95,0.95}
\definecolor{white}        {rgb}{1.00,1.00,1.00}
\newcommand{\textbk}[1]    {{\color{black}{#1}}}
\newcommand{\textbl}[1]    {{\color{blue}{#1}}}
\newcommand{\textaz}[1]    {{\color{azure}{#1}}}
\newcommand{\textcy}[1]    {{\color{cyan}{#1}}}
\newcommand{\textaq}[1]    {{\color{aqua}{#1}}}
\newcommand{\textgr}[1]    {{\color{green}{#1}}}
\newcommand{\textlm}[1]    {{\color{lime}{#1}}}
\newcommand{\textyw}[1]    {{\color{yellow}{#1}}}
\newcommand{\textor}[1]    {{\color{orange}{#1}}}
\newcommand{\textrd}[1]    {{\color{red}{#1}}}
\newcommand{\textpk}[1]    {{\color{pink}{#1}}}
\newcommand{\textmg}[1]    {{\color{magenta}{#1}}}
\newcommand{\textpu}[1]    {{\color{purple}{#1}}}
\newcommand{\textdb}[1]    {{\color{darkblue}{#1}}}
\newcommand{\textgy}[1]    {{\color{gray}{#1}}}
\newcommand{\textlg}[1]    {{\color{lightgray}{#1}}}
\newcommand{\textsh}[1]    {{\color{shade}{#1}}}
\newcommand{\textwh}[1]    {{\color{white}{#1}}}

\begin{document}
%  Header
\begin{minipage}{0.9\textwidth}
   \raggedright
   \large \textbf{\textsf{{\color{gray}EENG 533: Navigation Using GPS} \\
      Project 6: Calculation of Position from GPS Measurements}}
\end{minipage}
\vspace{1cm}

\section*{\textsf{Theory}}

\paragraph{Pseudoranges and Space Vehicle Positions}

The true distance, $\rho$, to a GPS satellite (also known as a space vehicle) is
the speed of light, $c$, times the difference of the true receive time,
$t_{rx}$, and the transmit time, $t_{sv}$ (the space vehicle's time):
\[ \rho = c\cdot \left( t_{rx} - t_{sv} \right) . \]
Unfortunately, both the space vehicle's time and the receiver's time have clock
errors ($\delta t_{sv}$ and $\delta t_{rx}$).  What we actually know are the
approximate times:
\begin{align*}
   \hat{t}_{sv} &= t_{sv} + \delta t_{sv} \\
   \hat{t}_{rx} &= t_{rx} + \delta t_{rx} .
\end{align*}
So, we do not get the true range from our measurements.  Rather, our GPS
receiver automatically calculates for us an apparent pseudorange, $\hat{\rho}$,
based on the receiver's approximate clock time, $\hat{t}_{rx}$, and the space
vehicle's approximate clock time, $\hat{t}_{sv}$, indicated in the GPS signal.
This pseudorange contains both the space vehicle's clock error, $\delta t_{sv}$,
as well as the receiver's clock error, $\delta t_{rx}$:
\[ \hat{\rho} = c\cdot \left( \hat{t}_{rx} - \hat{t}_{sv} \right) \]
\[ \to \hat{\rho} = c\cdot \left( t_{rx} + \delta t_{rx} \right)
   - c\cdot \left( t_{sv} + \delta t_{sv} \right) \]
such that the relationship to the true range is
\[ \rho = \hat{\rho} + c\; \delta t_{sv} - c\; \delta t_{rx} . \]

We can use the equation we learned on week 3 to estimate the space vehicle's
clock error:
\[ \delta t_{sv} = a_{f_0} + a_{f_1} \left( t_{sv} - t_{0_c} \right)
   + a_{f_2} \left( t_{sv} - t_{0_c} \right)^2 + \delta t_{rel} . \]
However, this is a function of the space vehicle's time, $t_{sv}$, which we do
not have yet.  So, we will solve this in a somwhat interative manner.  The first
thing we need to calculate is the approximate space vehicle time:
\[ \hat{t}_{sv} = \hat{t}_{rx} - \frac{\hat{\rho}}{c} . \]
Hint: this should be the first step in your \lstinline{calc_rx_pos} function!
Then, we will use this value as an approximation to $t_{sv}$ in the week 3
equation and we will disregard the relativistic correction term, $\delta
t_{rel}$, because we do not need this much accuracy yet:
\[ \delta t_{sv}' = a_{f_0} + a_{f_1} \left( \hat{t}_{sv} - t_{0_c} \right)
   + a_{f_2} \left( \hat{t}_{sv} - t_{0_c} \right)^2 . \]
(The coefficients $a_{f_0}$, $a_{f_1}$, $a_{f_2}$, and $t_{0_c}$ come from the
ephemeris data for each given space vehicle.)  Now we can get a better
approximation of the true space vehicle time:
\[ t_{sv}' = \hat{t}_{sv} - \delta t_{sv}' . \]

The \lstinline{calc_sv_pos} function will calculate more carefully the space
vehicle's clock error using the above time ($t_{sv}'$) and the receiver's
position.  But, since we do not yet know the receiver's position, we will use
the \lstinline{pos_rx} value sent to your \lstinline{calc_rx_pos} function as an
input.  This will be the output of your function from the previous time step
(epoch).  The \lstinline{project6_template.py} script will start this whole process with
the center of the Earth as the least biased guess:
\begin{lstlisting}
   pos_rx = np.array([0.0, 0.0, 0.0])
\end{lstlisting}
\noindent So, we just call the \lstinline{calc_sv_pos} function (only once per 
observation time epoch) as follows:
\begin{lstlisting}
   [pos_sv, dt_sv] = calc_sv_pos(ephem, t_sv_prime, pos_rx);
\end{lstlisting}
\noindent where \lstinline{pos_rx} is the receiver's position,
\lstinline{t_sv_prime} is the $t_{sv}'$ from above, \lstinline{pos_sv} is the
space vehicle's position, $[x_{sv}, y_{sv}, z_{sv}]$, and \lstinline{dt_sv} is
the space vehicle's clock error, $\delta t_{sv}$.  The \lstinline{calc_sv_pos}
function uses the receiver's position to adjust the space vehicle's position for
the rotation of the Earth during the propagation of the signal through space.
The poor guess of the receiver's position (the center of the Earth) will cause
some error for the first epoch.  But, after the first epoch, the receiver's
position will be better known and little error will remain.

If we are using a single frequency (which we are) we need to remove the group
delay correction:
\[ \delta t_{sv} = \delta t_{sv} - T_{GD} . \]
(This group delay value $T_{GD}$ comes from the ephemeris data for each given
space vehicle.)  The improved estimate of the space vehicle's clock error,
$\delta t_{sv}$, (\lstinline{dt_sv} above) can be used to get a better estimate
of the range, $\rho'$:
\[ \rho' = \hat{\rho} + c\; \delta t_{sv} . \]
This is still a pseudorange which has in it the receiver's clock error.  But,
this pseudorange is good enough because in the next step we will calculate
$\delta t_{rx}$ and the receiver's location using two things: the calculated
space vehicle's position, \lstinline{pos_sv}, and the pseudorange, $\rho'$.  So,
the results of this first part of your function should be a vector of
pseudoranges, $\vec{\rho}'$, and a matrix of positions, $[\vec{x}_{sv},\
\vec{y}_{sv},\ \vec{z}_{sv}]$.  (We notate vectors with underlines.)

\paragraph{Receiver Position and Clock Error}

If we have a set of pseudoranges, $\vec{\rho}'$, with the $\delta t_{sv}$ errors
removed and a set of corresponding space vehicle positions, $[\vec{x}_{sv},\
\vec{y}_{sv},\ \vec{z}_{sv}]$, we can iteratively calculate the position of the
receiver and the receiver's clock error which would be required for all those
pseudoranges to make sense.  We would use a guess for the receiver's position
and clock error to calculate what the pseudoranges to the space vehicles would
be, compare those pseudoranges to the $\vec{\rho}'$ values we already
calculated, and adjust our receiver's position and clock error accordingly.  We
do this until we begin to get consistent results.  This process is the
Taylor-series approximation that was explained in the lecture and is the same
concept as the Newton-Raphson technique.

First, we need to define a state vector.  It is the concatenation of two of the
function inputs: the receiver's position and the receiver's clock error, with
the speed of light factored in:
\[ \vec{x} = [ x_{rx} \ \ y_{rx} \ z_{rx} \ \ c\; \delta t_{rx} ]^\top . \]

There are two ways to get the pseudorange to a space vehicle.  The first is
time-based and is what we did in the previous step to get the $\rho'$ values.
The second is position-based and requires the positions of the space vehicles,
the position of the receiver, and the receiver's clock error:
\begin{equation}
   \vec{\tilde{\rho}} = \vec{r} + x_4
   \label{eq_rho_tilde}
\end{equation}
\[ \vec{r} = \sqrt{
   \left( \vec{x}_{sv} - x_1 \right)^2
      + \left( \vec{y}_{sv} - x_2 \right)^2
      + \left( \vec{z}_{sv} - x_3 \right)^2 } , \]
where $x_j$ is the $j$th element of $\vec{x}$.  Now, we can get the errors in
our distance-based pseudoranges:
\[ \delta \vec{\rho} = \vec{\rho}' - \vec{\tilde{\rho}} . \]

We need a way to relate these errors in pseudoranges to errors in the
$\vec{x}$ vector.  This relationship is the Jacobian matrix of
\eqref{eq_rho_tilde} with respect to the states of the state vector $\vec{x}$:
\[ \bm{H} = \begin{bmatrix}
      -\dfrac{\vec{x}_{sv} - x_1}{\vec{r}} &
      -\dfrac{\vec{y}_{sv} - x_2}{\vec{r}} &
      -\dfrac{\vec{z}_{sv} - x_3}{\vec{r}} & \vec{1}
   \end{bmatrix} . \]
The Jacobian relates the error in pseudoranges to the error in the state vector,
$\vec{x}$, this way:
\[ \delta \vec{\rho} = \bm{H} \delta \vec{x} . \]
So, to solve for the error in the state vector, we would use the pseudoinverse
of $\bm{H}$:
\[ \delta \vec{x} = (\bm{H}^\top \bm{H})^{-1} \bm{H}^\top
   \delta \vec{\rho} . \]
Be sure to use matrix multiplication operations (e.g. \lstinline{np.matmul()}, 
or the \lstinline{@} symbol).  Then, we can adjust our state vector with its error:
\[ \vec{x} = \vec{x} + \delta \vec{x} . \]

We repeat this loop, starting with equation \eqref{eq_rho_tilde}, while the
\lstinline{norm} of $\delta \vec{x}$ is greater than 10.  When we are done with
the loop, our receiver's position and clock error will be
\[ x_{rx} = x_1 \qquad
   y_{rx} = x_2 \qquad
   z_{rx} = x_3 \qquad
   \delta t_{rx} = \frac{x_4}{c} \]
and the residuals will be
\[ \vec{v} = \delta \vec{\rho} - \bm{H} \delta \vec{x} . \]

\section*{\textsf{Objectives}}

\begin{itemize}
   \item Develop an algorithm to perform single-point, single-frequency
      positioning using pseudorange measurements
   \item Understand some of the errors in your solution by comparing to a known
      location
\end{itemize}

\section*{\textsf{Collaboration}}

This is an individual laboratory.  You may discuss this lab with other students.
However, all source code that you generate and use and anything you turn in must
be your own.

\section*{\textsf{Task A: Write your \lstinline{calc_rx_pos} function}}

In this lab, you are given an hour of GPS RINEX observation data (\lstinline{*.YYo} file) and
the corresponding RINEX navigation (ephemeris) data (\lstinline{*.YYn} file).  Be sure 
to utilize the provided \lstinline{helper.py} and \lstinline{ephemeris.py} files as they have 
changed slightly from previous projects.  You will also need to utilize the previously 
developed \lstinline{calc_sv_pos} function.


The \lstinline{project6_template.py} script will load the observation and
ephemeris data.  It will iterate through each of the epochs of time in the
observation data, parsing out the PRNs and $\hat{\rho}$ values (L1 C/A-code
pseudoranges) from the observation data for the given epoch.  The script will
call your function.  Your function will then calculate the receiver's position
and clock error using single-point, single-frequency positioning.  It should
have the following interface:
\begin{lstlisting}
[pos_rx, dt_rx, v] = calc_rx_pos(prn_array, rho_hat_array, t_rx_hat, ephem_list, pos_rx, dt_rx)
\end{lstlisting}
where
\begin{center}
   \begin{tabular}{l|c|l}
      Variable & Size & Description \\
      \hline
      \texttt{prn\_array}         & $N$ & PRN values \\
      \texttt{rho\_hat\_array}    & $N$ & pseudoranges, $\hat{\rho}$ \\
      \texttt{t\_rx\_hat}  & $1$ & receiver time, $\hat{t}_{rx}$ \\
	\texttt{ephem\_list}  & $N$ & list of ephemeris objects \\
      \texttt{pos\_rx}     & $3$ & receiver position,
         $[x_{rx},\ y_{rx},\ z_{rx}]$ \\
      \texttt{dt\_rx}      & $1$ & receiver clock error,
         $\delta t_{rx}$ \\
      \texttt{v}           & $N$ & residuals, $\vec{v}$
   \end{tabular}
\end{center}
In the above table, $N$ is the number of space vehicles available during that
epoch to help locate the receiver.  Your code should follow these steps in order
to calculate the position and clock error of each space vehicle:
\begin{enumerate}
   \item Get $\hat{t}_{sv}$
   \item Get the ephemeris data for that space vehicle, estimate the clock error
      (without the relativistic correction term), and remove the clock error
      from $\hat{t}_{sv}$
   \item Call \lstinline{calc_sv_pos} to get the space vehicle's position and
      more accurate clock error
   \item For single-frequency positioning, remove the group delay
   \item Remove the effect of the space vehicle's clock error from its
      pseudorange
\end{enumerate}
For each space vehicle identified in the call to your function, you should now
have a pseudorange with $c\; \delta t_{sv}$ removed and a position for the space
vehicle in ECEF coordinates.  With the set of corrected pseudoranges and space
vehicle positions, follow the steps outlined in the \textbf{Theory} section to
get the receiver's position and single clock error.

The \lstinline{project6_template.py} script will continue from there to store the values
returned by your function.  It will print the true receiver
position and your calculated receiver position for the second time epoch.  The true
position (ECEF coordinates in meters and obtained from the online \href{https://geodesy.noaa.gov/OPUS/}
{OPUS tool}) is
\begin{lstlisting}
True Position = 
[  497796.51 -4884306.58  4058066.62]
\end{lstlisting}
There will be a discrepancy; that is expected.  To verify that your function is
working correctly, make sure your are getting exactly the following for the \textbf{second epoch}:
\begin{lstlisting}
pos_rx_array[1, :] = 
[  497794.82 -4884316.34  4058076.96]

v_array[1, :] = 
[-1.3    nan  5.42   nan   nan   nan -4.03 -1.22   nan   nan   nan   nan
 -0.66  1.04   nan   nan  1.24   nan -0.56   nan -2.54 -0.18   nan   nan
   nan   nan   nan  2.88   nan -0.08   nan   nan]
\end{lstlisting}
The second time epoch is chosen to avoid the problems associated with
initializing at the center of the earth for the first epoch.

Finally, the \lstinline{project6_template.py} script will convert your calculated
receiver position to an East, North, Up (ENU) local level frame.  It uses the
true position of the receiver as the origin.  So, these are position errors now.

\section*{\textsf{Task B: Evaluate accuracy of your algorithm}}

Towards the bottom of the \lstinline{project6_template.py} script, add plotting code and
comments.  Generate the following plots:
\begin{enumerate}
   \item A plot of the PRNs of the satellites used in the solution vs. time (sec of day).
      (Hint: plot PRNs vs. time using dots)
   \item ENU errors vs. time.  Plot all three as separate lines on the same
      axis.
   \item A horizontal map of the errors (East on x axis and North on y axis).
      This map should be scaled such that 1~m of error in the east direction is
      the same visible length as 1~m of error in the North direction.
   \item A plot of the receiver clock error vs. time
   \item A plot of the range residuals as a function of time.  (Hint: plot as
      individual points to avoid connecting lines)
\end{enumerate}

Make sure that all plots are correctly labeled.  That means give (1) a description of the
value and (2) units (use `ND' for non-dimensional).  As an example, a time axis
would be ``Time [s]''.  Answer questions such as, but not limited to the following:
\begin{itemize}
   \item What are the errors like?
   \item Is any axis worse than the others?
   \item How quickly do the errors change?
   \item Are there any anomalies (like jumps)?
   \item What would cause them?
\end{itemize}

\section*{\textsf{Deliverables}}

There are two files that you should upload to Canvas for your turn-in:
\begin{itemize}
   \item Your \lstinline{calc_rx_pos.py} function file
   \item Your modified \lstinline{project6_template_LASTNAME.py} script with code for plots and
      comments answering the questions
\end{itemize}

\section*{\textsf{Grading}}

You will be graded for coming up with the correct results (90\%) and your
analysis (10\%).  ``Correct results'' includes the correct generation of the
requested plots (not just that your \lstinline{calc_rx_pos} function works).

\end{document}
